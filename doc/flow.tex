\documentclass{article}

\usepackage{amsmath}


\begin{document}

    \section{DQMC review}
    In DQMC algorithm, we decomposite two-body term with
    \begin{equation}
        e^{-\Delta\tau U A^{2}} = \sum_{l} \gamma(l) e^{\sqrt{-\Delta\tau U} \eta(l) A}
    \end{equation}

    The partition funtion can be written as
    \begin{equation}
        \mathcal{Z} = Tr_{F} \sum_{C} \prod^{L_{trot}}_{\tau} e^{-\Delta\tau K} 
        \prod^{N_site}_{i} \gamma_{\tau, i} e^{\sqrt{-\Delta\tau U} \eta_{\tau, i} A_{i} }
    \end{equation}
    trace out the fermionic trace, (the const $\gamma_{i}$ should pull out of $Tr_{F}$)
    \begin{equation}
        \begin{aligned}
            \mathcal{Z} &= \sum_{C} \left[\prod^{L_{trot}} \prod^{N_{site}}_{i} \gamma_{\tau,i}\right]
            det(\mathcal{I} + \prod^{L_{trot}} e^{-\Delta\tau K}  \prod^{N_{site}}_{i} e^{\sqrt{-\Delta\tau U} \eta_{\tau, i} A_{i}})\\
            &= \sum_{C} \left[\prod^{L_{trot}} \prod^{N_{site}}_{i} \gamma_{\tau,i}\right]
            det(\mathcal{I} + \prod^{L_{trot}} B_{\tau}(C)) \\
            &= \sum_{C} W(C) D(C)
        \end{aligned}
    \end{equation}
    where $W(C) = \prod^{L_{trot}} \prod^{N_{site}}_{i} \gamma_{\tau,i}$, 
    $D(C)=det(\mathcal{I} + \prod^{L_{trot}} B_{\tau}(C))$ and
    $D(C)= Tr_{F} \prod^{L_{trot}} e^{-\Delta\tau K}  \prod^{N_{site}}_{i} e^{\sqrt{-\Delta\tau U} \eta_{\tau, i} A_{i}}$


    The finite temperature observables
    \begin{equation}
        \langle O \rangle = \frac{1}{\mathcal{Z}} Tr_{F} e^{-\beta H} O 
    \end{equation}
    we can also decomposite two-body term in numerator
    \begin{equation}
        \begin{aligned}
            \langle O \rangle &= \frac{1}{\mathcal{Z}} 
            Tr_{F} \sum_{C} \prod^{L_{trot}}_{\tau} e^{-\Delta\tau K} \prod^{N_{site}}_{i} \gamma_{\tau,i} e^{\sqrt{-\Delta\tau U} \eta_{\tau, i} A_{i}} O \\
            &= \sum_{C} W(C) Tr_{F} 
            \prod^{L_{trot}}_{\tau} e^{-\Delta\tau K} \prod^{N_{site}}_{i} e^{\sqrt{-\Delta\tau U} \eta_{\tau, i} A_{i}} O / 
            \sum_{C^{\prime}} W(C^{\prime}) D(C^{\prime}) \\
            &= \sum_{C} \left[ W(C)D(C)  \frac{Tr_{F} \prod^{L_{trot}}_{\tau} B_{\tau}(C) O }{D(C)} / \sum_{C^{\prime}} W(C^{\prime})D(C^{\prime}) \right] \\
            &= \sum_{C} S(C) \langle O \rangle_{C} / \sum_{C^{\prime}} S(C^{\prime})
        \end{aligned}
    \end{equation}
    where $S(C) = W(C)D(C)$ and $ \langle O \rangle_{C} = Tr_{F} \prod^{L_{trot}}_{\tau} B_{\tau}(C) O / Tr_{F} \prod^{L_{trot}}_{\tau} B_{\tau}(C)$.

    \section{Sign Problem}
    We can rewrite the observables as
    \begin{equation}
        \begin{aligned}
            \langle O \rangle 
            &= \sum_{C} S(C) \langle O \rangle_{C} / \sum_{C^{\prime}} S(C^{\prime})\\
            &= \frac{ \sum_{C} \vert S(C) \vert \frac{S(C)}{\vert S(C) \vert} \langle O \rangle_{C} / \sum_{C^{\prime\prime}} \vert S(C^{\prime\prime}) \vert }
            {\sum_{C^{\prime}} \vert S(C) \vert \frac{S(C)}{\vert S(C) \vert} / \sum_{C^{\prime\prime\prime}} \vert S(C^{\prime\prime\prime}) \vert} 
        \end{aligned}
    \end{equation}
    where $S(C) = W(C)D(C)$ and the partition function $Z = \sum_{C} S(C)$.
    We can calculate the denominator and numerator in Eq.(6) by Monte Carlo algorithm.
    If the denominator goes to zero, the Sign Problem occurs.

    \section{divide the sampling}

    The steps in this section may be confused, we first explain the main idea of this procedure.
    \begin{itemize}
        \item First, we introduce an artificial cutoff $\lambda$ in the observables, $\langle O \rangle^{\lambda}$,
        the observables recover the exact one when $\lambda \rightarrow -\infty$, $\langle O \rangle = \langle O \rangle^{-\infty}$.
        \item Second, we find a relation between exact one and cutoff one $\langle O \rangle^{-\infty} = \Omega + \langle O \rangle^{\lambda} / (\omega+1)$
        \item Third, rescaling the observables, $\langle O \rangle \leftarrow \langle O \rangle^{\lambda} $, and repeat the second untill the results are solvable.
    \end{itemize}
    This three procedure form a renormalization process. At the end, we reverse these steps,
    so we start from a solvable results, and recursively find the origin exact results.    

    We write the observables
    \begin{equation}
        \begin{aligned}
            \langle O \rangle &= \frac{ \sum_{C} S(C) \langle O \rangle_{C} }{ \sum_{C^{\prime}} S(C^{\prime}) }\\
            &=\frac{ \sum_{C} ReS(C) \frac{S(C)}{ReS(C)} \langle O \rangle_{C} }{ \sum_{C^{\prime}} ReS(C^{\prime}) }
        \end{aligned}
    \end{equation}
    The second line is true because the partition function is real, and $Re$ means real part.
    Before we reweight the denominator and numerator, we first divide the sampling space.
    For example, we can divide the denominator
    \begin{equation}
        \sum_{C^{\prime}} ReS(C^{\prime}) = \sum_{C^{\prime}} \Theta(\lambda-ReS(C^{\prime})) ReS(C^{\prime}) +  \sum_{C^{\prime}} \Theta(ReS(C^{\prime})-\lambda) ReS(C^{\prime})
    \end{equation}
    For notation convenience, we define
    \begin{equation}
        \begin{aligned}
            \sum^{>\lambda}_{C} &= \sum_{C} \Theta(ReS(C)-\lambda)\\
            \sum^{\mu<<\nu}_{C} &=  \sum_{C} \Theta(ReS(C)-\mu) \Theta(\nu-ReS(C))
        \end{aligned}
    \end{equation}
    The first one means the summation of these HS configurations which have $ReS(C) > \lambda$
    and the latter one means the summation of these HS configurations whose $ReS(C)$ lying in a shell,
    $\mu< ReS(C) <\nu$.

    Next, we devide the numerator and denominator,
    \begin{equation}
        \begin{aligned}
            \langle O \rangle &=\frac{ \sum_{C} ReS(C) \frac{S(C)}{ReS(C)} \langle O \rangle_{C} }{ \sum_{C^{\prime}} ReS(C^{\prime}) }\\
            &=\frac{\sum^{-\infty<<\lambda_{1}}_{C_{1}} ReS(C_{1}) \frac{S(C_{1})}{ReS(C_{1})} \langle O \rangle_{C_{1}} + \sum^{>\lambda_{1}}_{C_{2}} ReS(C_{2}) \frac{S(C_{2})}{ReS(C_{2})} \langle O \rangle_{C_{2}}}
            {\sum^{-\infty<<\lambda_{1}}_{C_{3}} ReS(C_{3}) + \sum^{>\lambda_{1}}_{C_{4}} ReS(C_{4})} \\
            %&=\frac{\left(\sum^{-\infty<<\lambda_{1}}_{C} ReS(C) \frac{S(C)}{ReS(C)} \langle O \rangle_{C} + \sum^{>\lambda_{1}}_{C} ReS(C) \frac{S(C)}{ReS(C)} \langle O \rangle_{C}\right) / \sum^{>\lambda_{1}}_{C} ReS(C)}
            %{1+\sum^{-\infty<<\lambda_{1}}_{C} ReS(C) /  \sum^{>\lambda_{1}}_{C} ReS(C)}
        \end{aligned}
    \end{equation}
    dividing $\sum^{>\lambda_{1}}_{C} ReS(C)$ in the numerator and denominator simultaneously, the denominator will be
    \begin{equation}
        \begin{aligned}
            &\frac{\sum^{-\infty<<\lambda_{1}}_{C_{1}} ReS(C_{1}) + \sum^{>\lambda_{1}}_{C_{2}} ReS(C_{2})}{\sum^{>\lambda_{1}}_{C_{3}} ReS(C_{3})}\\
            & = \frac{\sum^{-\infty<<\lambda_{1}}_{C_{1}} ReS(C_{1})}{\sum^{>\lambda_{1}}_{C_{2}} ReS(C_{2})} + 1\\
            & = \omega(-\infty,\lambda) + 1
        \end{aligned}
    \end{equation}
    Here we define $\omega(\mu,\nu) = \frac{\sum^{\mu<<\nu}_{C_{1}} ReS(C_{1})}{\sum^{>\nu}_{C_{2}} ReS(C_{2})}$
    And now, we look at the numerator, the first term
    \begin{equation}
        \begin{aligned}
            &\frac{\sum^{-\infty<<\lambda_{1}}_{C_{1}} ReS(C_{1}) \frac{S(C_{1})}{ReS(C_{1})} \langle O \rangle_{C_{1}}}{\sum^{>\lambda}_{C_{2}} ReS(C_{2})} \\
            &=\frac{\sum^{-\infty<<\lambda_{1}}_{C_{1}} ReS(C_{1}) \frac{S(C_{1})}{ReS(C_{1})} \langle O \rangle_{C_{1}} / \sum^{-\infty<<\lambda_{1}}_{C_{3}} ReS(C_{3})}{\sum^{>\lambda}_{C_{2}} ReS(C_{2}) / \sum^{-\infty<<\lambda_{1}}_{C_{4}}ReS(C_{4})} \\
            &=\langle O \rangle^{-\infty<<\lambda_{1}} \omega(-\infty, \lambda_{1})
        \end{aligned}
    \end{equation}
    the second term
    \begin{equation}
        \frac{\sum^{>\lambda_{1}}_{C_{1}} ReS(C_{1}) \frac{S(C_{1})}{ReS(C_{1})} \langle O \rangle_{C_{1}}}{\sum^{>\lambda}_{C_{2}} ReS(C_{2})} = \langle O \rangle^{>\lambda_{1}}
    \end{equation}
    Here, we define $\langle O \rangle^{>\lambda}$ and $\langle O \rangle^{\nu<<\mu}$.
    The observables now can be written
    \begin{equation}
    \boxed{
        \langle O \rangle = \frac{\langle O \rangle^{-\infty<<\lambda_{1}} \omega(-\infty, \lambda_{1}) + \langle O \rangle^{>\lambda_{1}}}{\omega(-\infty,\lambda) + 1}
    }
    \end{equation}
    
    Now comes to the critical part, by choosing a proper $\lambda_{1}$,
    the denominator $\omega(-\infty,\lambda) + 1$ \textbf{will not} goes to zero when Sign Problem occurs,
    and the Sign Problem will be absorbed in the $\langle O \rangle^{>\lambda_{1}}$ and $\omega(-\infty,\lambda)$.
    For example, we can choose a negative $\lambda_{1}$ with large absolute value, 
    and the term of $\langle O \rangle^{-\infty<<\lambda_{1}}$ can be compute by
    \begin{equation}
        \begin{aligned}
            &\langle O \rangle^{-\infty<<\lambda_{1}} = \frac{\sum^{-\infty<<\lambda_{1}}_{C_{1}} ReS(C_{1})  \frac{S(C_{1})}{ReS(C_{1})} \langle O \rangle_{C_{1}} }{\sum^{-\infty<<\lambda_{1}}_{C_{2}} ReS(C_{2})}\\
            &=\frac{\sum^{-\infty<<\lambda_{1}}_{C_{1}} \vert ReS(C_{1}) \vert \frac{ReS(C_{1})}{\vert ReS(C_{1}) \vert} \frac{S(C_{1})}{ReS(C_{1})} \langle O \rangle_{C_{1}} / \sum^{-\infty<<\lambda1}_{C_{3}} \vert ReS(C_{3}) \vert }{\sum^{-\infty<<\lambda_{1}}_{C_{2}} \vert ReS(C_{2}) \vert \frac{ReS(C_{2})}{\vert ReS(C_{2}) \vert} / \sum^{-\infty<<\lambda1}_{C_{4}} \vert ReS(C_{4}) \vert}\\
        \end{aligned}
    \end{equation}
    after reweight, this value is well behaved, because the $ReS(C)$ lie in a shell, the average sign equals to $1$.

    The $\langle O \rangle^{>\lambda_{1}}$ has the form
    \begin{equation}
        \langle O \rangle^{>\lambda_{1}} = \frac{\sum^{>\lambda_{1}}_{C_{1}} ReS(C_{1}) \frac{S(C_{1})}{ReS(C_{1})} \langle O \rangle_{C_{1}}}{\sum^{>\lambda_{1}}_{C_{2}} ReS(C_{2})}
    \end{equation}
    this is \textbf{not} well behaved since the denominator may be close to zero if the shell is narrow.
    this happend also in $\omega(-\infty, \lambda_{1})$
    \begin{equation}
        \omega(\mu,\nu) = \frac{\sum^{\mu<<\nu}_{C_{1}} ReS(C_{1})}{\sum^{>\nu}_{C_{2}} ReS(C_{2})}
    \end{equation}

    Then we divide the sampling recursively, we have this equation similiar with Eq.(14)
    \begin{equation}
        \begin{aligned}
            &\langle O \rangle^{>\lambda_{1}} = \frac{\sum^{>\lambda_{1}}_{C_{1}} ReS(C_{1}) \frac{S(C_{1})}{ReS(C_{1})} \langle O \rangle_{C_{1}}}{\sum^{>\lambda_{1}}_{C_{2}} ReS(C_{2})}\\
            &=\frac{\sum^{\lambda_{1}<<\lambda_{2}}_{C_{1}} ReS(C_{1}) \frac{S(C_{1})}{ReS(C_{1})} \langle O \rangle_{C_{1}} + \sum^{>\lambda_{2}}_{C_{2}} ReS(C_{2}) \frac{S(C_{2})}{ReS(C_{2})} \langle O \rangle_{C_{2}}  }
            {\sum^{\lambda_{1}<<\lambda_{2}}_{C_{3}} ReS(C_{3}) + \sum^{>\lambda_{2}}_{C_{4}} ReS(C_{4})}\\
            &=\frac{\omega(\lambda_{1},\lambda_{2}) \langle O \rangle^{\lambda_{1}<<\lambda_{2}} + \langle O \rangle^{>\lambda_{2}}}{\omega(\lambda_{1}, \lambda_{2})+1}
        \end{aligned}
    \end{equation}

    Similiar to this, the $\omega(-\infty,\lambda_{1})$ can be written as
    \begin{equation}
        \begin{aligned}
            \omega(-\infty,\lambda_{1}) &= \frac{\sum^{-\infty<<\lambda_{1}}_{C_{1}} ReS(C_{1})}{\sum^{>\lambda_{1}}_{C_{2}} ReS(C_{2})} \\
            &= \frac{\sum^{-\infty<<\lambda_{1}}_{C_{1}} ReS(C_{1})}{\sum^{\lambda_{1}<<\lambda_{2}}_{C_{2}} ReS(C_{2}) + \sum^{>\lambda_{2}}_{C_{3}} ReS(C_{3})} \\
            &= \frac{\sum^{-\infty<<\lambda_{1}}_{C_{1}} ReS(C_{1}) / \sum^{\lambda_{1}<<\lambda_{2}}_{C_{2}}ReS(C_{2})}
            {1+\sum^{>\lambda_{2}}_{C_{3}} ReS(C_{3})/\sum^{\lambda_{1}<<\lambda_{2}}_{C_{4}}ReS(C_{4})} \\
            &= \frac{\omega^{\prime}(-\infty, \lambda_{1}, \lambda_{2})}
            {1+1/\omega(\lambda_{1}, \lambda_{2})} \\
            &= \frac{\omega^{\prime}(-\infty, \lambda_{1}, \lambda_{2}) \omega(\lambda_{1}, \lambda_{2})}{ \omega(\lambda_{1}, \lambda_{2}) + 1}
        \end{aligned}
    \end{equation}

    Here we define the 
    \begin{equation}
        \begin{aligned}
            \omega^{\prime}(\mu, \nu, \xi) &= \frac{ \sum^{\mu<<\nu}_{C_{1}} ReS(C_{1}) }
            { \sum^{\nu<<\xi}_{C_{2}}ReS(C_{2}) } \\
            &= \frac{ \sum^{\mu<<\xi}_{C_{1}} \vert ReS(C_{1}) \vert \frac{ReS(C_{1})}{\vert ReS(C_{1}) \vert}  \Theta(\nu - ReS(C_{1})) / \sum^{\mu<<\xi}_{C_{2}} \vert ReS(C_{2}) \vert}
            { \sum^{\mu<<\xi}_{C_{3}} \vert ReS(C_{3}) \vert \frac{ReS(C_{3})}{\vert ReS(C_{3}) \vert}  \Theta(ReS(C_{3})-\nu) / \sum^{\mu<<\xi}_{C_{4}} \vert ReS(C_{4}) \vert} 
        \end{aligned}
    \end{equation}
    This is also well behaved since $\mu$ and $\xi$ are close.
    
    
    We can \textbf{recursively repeat this procedure}, and every observables in the narrow shell $\langle O \rangle^{\lambda_{n}<<\lambda_{n+1}}$ is well
    behaved, untill we reach a boundary, that is $\lambda_{M} = 0$. When we reach this boundary, the higher observables is
    \begin{equation}
        \begin{aligned}
            \langle O \rangle^{>\lambda_{M}} &= \langle O \rangle^{>0} \\
            &=\frac{\sum_{C_{1}} \Theta(ReS(C_{1})) ReS(C_{1}) \frac{S(C_{1})}{ReS(C_{1})} \langle O \rangle_{C_{1}}}{\sum_{C_{2}} \Theta(ReS(C_{2})) ReS(C_{2})}\\
            &=\frac{\sum^{>0}_{C_{1}} \vert ReS(C_{1}) \vert \frac{ReS(C_{1})}{\vert ReS(C_{1}) \vert}  \frac{S(C_{1})}{ReS(C_{1})} \langle O \rangle_{C_{1}} / \sum^{>0}_{C_{3}} \vert ReS(C_{3}) \vert}
            {\sum^{>0}_{C_{2}} \vert ReS(C_{2}) \vert \frac{ReS(C_{2})}{\vert ReS(C_{2}) \vert} / \sum^{>0}_{C_{4}} \vert ReS(4) \vert}
        \end{aligned}
    \end{equation}
    In fact, the reweight in the third line can be ignored, because the sign always positive.
    The $\omega(\lambda_{M-1}, \lambda_{M})$ can be written as
    \begin{equation}
        \begin{aligned}
            &\omega(\lambda_{M-1}, \lambda_{M}) = \omega(\lambda_{M-1}, 0)\\
            &= \frac{\sum^{\lambda_{M-1}<<0}_{C_{1}} ReS(C_{1})}
            {\sum^{>0}_{C_{2}} ReS(C_{2})} \\
            &= \frac{\sum^{\lambda_{M-1}<<\infty}_{C_{1}} \vert ReS(C_{1}) \vert \frac{ReS(C_{1})}{\vert ReS(C_{1})\vert } \Theta(-ReS(C_{1}))  / \sum^{\lambda_{M-1}<<\infty}_{C_{2}} \vert ReS(C_{2}) \vert}
            {\sum^{\lambda_{M-1}<<\infty}_{C_{3}} \vert ReS(C_{3}) \vert \frac{ReS(C_{3})}{\vert ReS(C_{3})\vert } \Theta(ReS(C_{1}))  / \sum^{\lambda_{M-1}<<\infty}_{C_{4}} \vert ReS(C_{4}) \vert}
        \end{aligned}
    \end{equation}
    This value is well behaved because of the $\Theta$ funtion, unless the 
    $\sum^{\lambda_{M-1}<<\infty}_{C_{4}} \vert ReS(C_{4}) \vert$ is much larger than $\sum^{>0}_{C_{2}} ReS(C_{2})$.
    However, we can choosing the value of $\lambda_{M-1}$ to avoid.

    Now, we have
    \begin{equation}
    \boxed{
        \langle O \rangle^{\lambda_{M}=0} = \frac{\sum^{>0}_{C_{1}} \vert ReS(C_{1}) \vert \frac{ReS(C_{1})}{\vert ReS(C_{1}) \vert}  \frac{S(C_{1})}{ReS(C_{1})} \langle O \rangle_{C_{1}} / \sum^{>0}_{C_{3}} \vert ReS(C_{3}) \vert}
        {\sum^{>0}_{C_{2}} \vert ReS(C_{2}) \vert \frac{ReS(C_{2})}{\vert ReS(C_{2}) \vert} / \sum^{>0}_{C_{4}} \vert ReS(4) \vert}
        }
    \end{equation}
    \begin{equation}
        \boxed{
        \omega(\lambda_{M-1}, \lambda_{M}=0) = \frac{\sum^{\lambda_{M-1}<<\infty}_{C_{1}} ReS(C_{1}) \Theta(-ReS(C_{1}))  }
        {\sum^{\lambda_{M-1}<<\infty}_{C_{3}} ReS(C_{3}) \Theta(ReS(C_{1}))  }
        }
    \end{equation}
    And we have the recursive relation
    \begin{equation}
        \boxed{
        \omega(\lambda_{N-1},\lambda_{N}) = \frac{\omega^{\prime}(\lambda_{N-1}, \lambda_{N}, \lambda_{N+1}) \omega(\lambda_{N}, \lambda_{N+1})}{ \omega(\lambda_{N}, \lambda_{N+1}) + 1}
        }
    \end{equation}
    and
    \begin{equation}
        \boxed{
        \langle O \rangle^{>\lambda_{N-1}} = \frac{\omega(\lambda_{N-1},\lambda_{N}) \langle O \rangle^{\lambda_{N-1}<<\lambda_{N}} + \langle O \rangle^{>\lambda_{N}}}{\omega(\lambda_{N-1}, \lambda_{N})+1}
        }
    \end{equation}
    where
    \begin{equation}
        \boxed{
        \omega^{\prime}(\lambda_{N-1}, \lambda_{N}, \lambda_{N+1}) = \frac{ \sum^{\lambda_{N-1}<<\lambda_{N+1}}_{C_{1}} ReS(C_{1})  \Theta(\lambda_{N} - ReS(C_{1}))}
        { \sum^{\lambda_{N-1}<<\lambda_{N+1}}_{C_{3}} ReS(C_{3})  \Theta(ReS(C_{3})-\lambda_{N}) } 
        }
    \end{equation}
    and 
    \begin{equation}
        \boxed{
            \langle O \rangle^{\lambda_{N-1}<<\lambda_{N}} = \frac{\sum^{\lambda_{N-1}<<\lambda_{N}}_{C_{1}} ReS(C_{1})  \frac{S(C_{1})}{ReS(C_{1})} \langle O \rangle_{C_{1}}  }
            {\sum^{\lambda_{N-1}<<\lambda_{N}}_{C_{2}} ReS(C_{2}) }\\
        }
    \end{equation}

    We can start from $\lambda_{M}=0$, $\lambda_{M-1}$, $\lambda_{M-2}$, and repeat the recursion, which is the inverse of the origin recursion,
    to get the $\langle O \rangle = \langle O \rangle^{>-\infty}$ finally.


    \section{flow equation}
    In continumm limit
    \begin{equation}
        \begin{aligned}
            S^{>\lambda_{1}} &= (1 + \omega (\lambda_{2})* d \lambda) S^{>\lambda_{2}}\\
            d S^{\lambda_{2}} &= \omega (\lambda_{2})*d \lambda \\
            S^{\lambda_{2}} & = S^{>0} e^{\int^{\lambda_{2}}_{0} \omega(\lambda) d\lambda}
        \end{aligned}
    \end{equation}
    observables
    \begin{equation}
        \begin{aligned}
            \langle O \rangle^{>\lambda_{1}} - \langle O \rangle^{>\lambda_{2}} &= 
            \frac{\omega(\lambda_{1}, \lambda_{2})}{1+\omega(\lambda_{1},\lambda_{2})} (\langle O \rangle^{\lambda_{1}<<\lambda_{2}} - \langle O \rangle^{>\lambda_{2}} ) \\
            d \langle O \rangle^{\lambda_{2}} &= \omega(\lambda_{2}) d \lambda (\langle O \rangle^{\lambda_{1}<<\lambda_{2}} - \langle O \rangle^{>\lambda_{2}} ) \\
            \langle O \rangle^{\lambda_{2}} & = \langle O \rangle^{>0} e^{-\int^{\lambda_{2}}_{0} \omega(\lambda) d\lambda} + \langle O \rangle^{-\infty<<\lambda_{2}}
        \end{aligned}
    \end{equation}

\end{document}

